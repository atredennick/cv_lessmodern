%-------------------------------------------------------------------------------
%	SECTION TITLE
%-------------------------------------------------------------------------------
\cvsection{Publications}


%-------------------------------------------------------------------------------
%	CONTENT
%-------------------------------------------------------------------------------
\begin{pubentries}
\begin{\small}
Wilcox*, K.R., {\bf{A.T. Tredennick}}*, S. Koerner, E. Grman, L. Hallett, M. Avolio, K. La Pierre, G. Houseman, F. Isbell, D. Johnson, J. Alatalo, A. Baldwin, E. Bork, E. Boughton, W. Bowman, A. Britton, J. Cahill, S. Collins, G-Z. Du, A. Eskelinen, L. Gough, A. Jentsch, C. Kern, K. Klanderud, A. Knapp, J. Kreyling, Y. Luo, J. McLaren, P. Megonigal, V. Onipchenko, J. Prevéy, J. Price, C. Robinson, O. Sala, M. Smith, N. Soudzilovskaia, L. Souza, D. Tilman, S. White, Z. Xu, L. Yahdjian, Q. Yu, P. Zhang, Y, Zhang. (2017). Asynchrony among local communities stabilizes ecosystem function of metacommunities. \emph{Ecology Letters} 20(12):1534–1545.\\
{\footnotesize*{\emph{Shared first authorship.}}}\\
\\
{\bf{Tredennick, A.T.}}, P.B. Adler, \& F.R. Adler. (2017). The relationship between species richness and ecosystem variability is shaped by the mechanism of coexistence. \emph{Ecology Letters} 20(8):958-968. \\
\\
{\bf{Tredennick, A.T.}}, M.B. Hooten, \& P.B. Adler. (2017). Do we need demographic data to forecast plant population dynamics? \emph{Methods in Ecology \& Evolution} 8(5):541-551.\\
\\
{\bf{Tredennick, A.T.}}, C. de Mazancourt, M. Loreau, \& P.B. Adler. (2017). Environmental responses, not species interactions, determine synchrony of dominant species in semiarid grasslands. \emph{Ecology} 98(4):971-981.\\
\\
Kulmatiski, A., P.B. Adler, J.M. Stark, \& {\bf{A.T. Tredennick.}} (2017). Water and nitrogen uptake are better associated with resource availability than root biomass. \emph{Ecosphere} 8(3):e01738.\\
\\
{\bf{Tredennick, A.T.}}, M.B. Hooten, C.L. Aldridge, C. Homer, A.R. Kleinhesselink, \& P.B. Adler. (2016). Forecasting climate change impacts on plant populations over large spatial extents. \emph{Ecosphere} 7(10):e01525.\\
\\
{\bf{Tredennick, A.T.}}, P.B. Adler, J.B. Grace, W.S. Harpole, E.T. Borer, E.W. Seabloom, \& 36 co-authors. (2016). Comment on “Worldwide evidence of a unimodal relationship between productivity and plant species richness.” \emph{Science} 35(6272):457a-457c.\\
\\
{\bf{Tredennick, A.T.}}, M. Karemb\'{e}, F. Demb\'{e}l\'{e}, J. Dohn, \& N.P Hanan. (2015). No effects of fire, large herbivores, and their interaction on regrowth of harvested trees in two West African savannas. \emph{African Journal of Ecology} 53(4):487-495.\\
\\
Hanan, N.P., {\bf{A.T. Tredennick}}, L. Prihodko, G. Bucini, \& J. Dohn. (2015). Analysis of stable states in global savannas – a response to Staver and Hansen. \emph{Global Ecology and Biogeography} 24(8):988-989.\\
\\
{\bf{Tredennick, A.T.}} \& N.P. Hanan. (2015). Effects of tree harvest on the stable-state dynamics of savanna and forest. \emph{The American Naturalist} 5(185):E153-E165.\\
\\
Hanan, N.P., {\bf{A.T. Tredennick}}, L. Prihodko, G. Bucini, \& J. Dohn. (2014). Analysis of stable states in global savannas: Is the CART pulling the horse? \emph{Global Ecology and Biogeography} 23(3):259-263.\\
\\
{\bf{Tredennick, A.T.}}, L.P. Bentley, \& N.P. Hanan. (2013). Allometric convergence in savanna trees and implications for the use of plant scaling models in variable ecosystems. \emph{PLoS One} 8(3):e58241.\\
\\
Rice, J., {\bf{A.T. Tredennick}}, \& L. Joyce. (2011). The climate of the Shoshone National Forest: A synthesis of past changes, future projections, and ecosystem implications. USFS General Technical Report No. 264.\\
\\
Sutton, A.E., J. Dohn, K. Loyd, {\bf{A.T. Tredennick}}, G. Bucini, A. Solrzano, L. Prihodko, \& N.P. Hanan. (2010). Letter: Does warming increase the risk of civil war in Africa? \emph{Proceedings of the National Academy of Sciences} 107(25):E102.
\end{small}

%---------------------------------------------------------
\end{pubentries}
